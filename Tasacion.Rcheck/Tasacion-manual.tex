\nonstopmode{}
\documentclass[a4paper]{book}
\usepackage[times,inconsolata,hyper]{Rd}
\usepackage{makeidx}
\usepackage[utf8,latin1]{inputenc}
% \usepackage{graphicx} % @USE GRAPHICX@
\makeindex{}
\begin{document}
\chapter*{}
\begin{center}
{\textbf{\huge Package `Tasacion'}}
\par\bigskip{\large \today}
\end{center}
\begin{description}
\raggedright{}
\inputencoding{latin1}
\item[Type]\AsIs{Package}
\item[Title]\AsIs{Genera una tasacion}
\item[Version]\AsIs{1.2}
\item[Date]\AsIs{2015-07-23}
\item[Author]\AsIs{Federico Munyoz }\email{federico.munyoz@innocv.com}\AsIs{}
\item[Maintainer]\AsIs{Federico Munyoz }\email{federico.munyoz@innocv.com}\AsIs{}
\item[Description]\AsIs{Genera una tasacion y la calidad a partir del caso a estimar, los parámetros del modelo y una muestra.}
\item[License]\AsIs{---}
\item[Encoding]\AsIs{latin1}
\item[NeedsCompilation]\AsIs{no}
\end{description}
\Rdcontents{\R{} topics documented:}
\inputencoding{utf8}
\HeaderA{Tasacion-package}{Genera una tasacion}{Tasacion.Rdash.package}
\aliasA{Tasacion}{Tasacion-package}{Tasacion}
\keyword{package}{Tasacion-package}
%
\begin{Description}\relax
Genera una tasacion a partir de 3 archivos csv ubicados en el directorio indicado:
- El caso a estimar (CasoEstimacion.csv)
- Los parametros del modelo (ParametrosCalculo.csv)
- La muestra de tasaciones de la zona (DatosMuestra.csv)
\end{Description}
%
\begin{Details}\relax

\Tabular{ll}{
Package: & Tasacion\\{}
Type: & Package\\{}
Version: & 1.0\\{}
Date: & 2015-03-24\\{}
License: & ---\\{}
}
Genera una tasacion a partir de un caso a estimar, unos parametros del modelo y una muestra.
\end{Details}
%
\begin{Author}\relax
Federico Munyoz

Maintainer: Federico Munyoz <federico.munyoz@innocv.com>
\end{Author}
\inputencoding{utf8}
\HeaderA{CorreccionPorOtrasVariables}{CorreccionPorOtrasVariables}{CorreccionPorOtrasVariables}
%
\begin{Description}\relax
Calcula el factor de correccion por otras variables
\end{Description}
%
\begin{Usage}
\begin{verbatim}
CorreccionPorOtrasVariables(VariablesDeLaMuestra, CasoEstimacion, Parametros)
\end{verbatim}
\end{Usage}
%
\begin{Arguments}
\begin{ldescription}
\item[\code{VariablesDeLaMuestra}] Matriz de variables de la muestra
\item[\code{CasoEstimacion}] Caso a estimar
\item[\code{Parametros}] Parametros de calculo del modelo
\end{ldescription}
\end{Arguments}
%
\begin{Value}
Devuelve los valores de correccion para cada muestra
\end{Value}
\inputencoding{utf8}
\HeaderA{GrabarTasacion}{Graba la tasacion en un archivo Tasacion.csv}{GrabarTasacion}
%
\begin{Description}\relax
A partir de la muestra con los parametros calculados procede a calcular la tasacion y grabar el archivo csv.
\end{Description}
%
\begin{Usage}
\begin{verbatim}
GrabarTasacion(Directorio, Muestra)
\end{verbatim}
\end{Usage}
%
\begin{Arguments}
\begin{ldescription}
\item[\code{Directorio}] Directorio donde se grabara el archivo Tasacion.csv
\item[\code{Muestra}] Muestra con las columnas Pesos, PesosXwCorregido y PesosXwCorregidoCuadrado para realizar la estimacion
\end{ldescription}
\end{Arguments}
%
\begin{Value}
Devuelve el archivo Tasacion.csv
\end{Value}
\inputencoding{utf8}
\HeaderA{LeerArchivos}{Leer archivos con los datos para estimar la muestra}{LeerArchivos}
%
\begin{Description}\relax
Lee los archivos CasoEstimacion.csv, ParametrosCalculo.csv y DatosMuestra.csv en la ubicacion indicada.
\end{Description}
%
\begin{Usage}
\begin{verbatim}
LeerArchivos(Directorio)
\end{verbatim}
\end{Usage}
%
\begin{Arguments}
\begin{ldescription}
\item[\code{Directorio}] Directorio donde estan ubicados los archivos
\end{ldescription}
\end{Arguments}
\inputencoding{utf8}
\HeaderA{ObtieneDatosDelModelo}{Obtiene los coeficientes del modelo a partir de la tipologia a estimar y los parametros}{ObtieneDatosDelModelo}
%
\begin{Description}\relax
Obtiene los coeficientes del modelo a partir de la tipologia a estimar y los parametros
\end{Description}
%
\begin{Usage}
\begin{verbatim}
ObtieneDatosDelModelo(ParametrosDelModelo, Tipologia)
\end{verbatim}
\end{Usage}
%
\begin{Arguments}
\begin{ldescription}
\item[\code{ParametrosDelModelo}] Parametros del modelo para los calculos
\item[\code{Tipologia}] Tipologia del caso a estimar
\end{ldescription}
\end{Arguments}
\inputencoding{utf8}
\HeaderA{ObtieneOtrasVariables}{Obtiene otras variables a partir de la muestra}{ObtieneOtrasVariables}
%
\begin{Description}\relax
Obtiene las siguientes variables de la muestra:  SinAscensor, EsPlantaBaja, EsSemiSotano, EsInterior, EsAtico, NDePlanta
\end{Description}
%
\begin{Usage}
\begin{verbatim}
ObtieneOtrasVariables(MatrizOtrasVariables)
\end{verbatim}
\end{Usage}
%
\begin{Arguments}
\begin{ldescription}
\item[\code{MatrizOtrasVariables}] Muestra con los datos de otras variables
\end{ldescription}
\end{Arguments}
\inputencoding{utf8}
\HeaderA{SeleccionarOtrasVariables}{Selecciona las variables NumeroDeAscensores, ExteriorInterior, NumeroDePlanta, y DatosDeLaPlanta de la muestra recibida}{SeleccionarOtrasVariables}
%
\begin{Usage}
\begin{verbatim}
SeleccionarOtrasVariables(Muestra)
\end{verbatim}
\end{Usage}
%
\begin{Arguments}
\begin{ldescription}
\item[\code{Muestra}] Muestra de la que extraer las columnas NumeroDeAscensores, ExteriorInterior, NumeroDePlanta, y DatosDeLaPlanta

\end{ldescription}
\end{Arguments}
\inputencoding{utf8}
\HeaderA{SeleccionarVariables}{Selecciona las variables ValorUnitarioDeMercado, NumeroDeBanos, SuperficieDeLaParcela, SuperficieConstruidaDeLaVivienda, CosteDeConstruccionBrutoUnitarioVivienda, PorcentajeDepreciacionVivienda y NumeroDeDormitorios de la muestra. Dependiendo de la tipologia del caso a estimar devuelve la SuperficieDeLaParcela o el NumeroDeDormitorios.}{SeleccionarVariables}
%
\begin{Description}\relax
Selecciona las variables ValorUnitarioDeMercado, NumeroDeBanos, SuperficieDeLaParcela, SuperficieConstruidaDeLaVivienda, CosteDeConstruccionBrutoUnitarioVivienda, PorcentajeDepreciacionVivienda y NumeroDeDormitorios de la muestra. Dependiendo de la tipologia del caso a estimar devuelve la SuperficieDeLaParcela o el NumeroDeDormitorios.
\end{Description}
%
\begin{Usage}
\begin{verbatim}
SeleccionarVariables(Muestra, Tipologia)
\end{verbatim}
\end{Usage}
%
\begin{Arguments}
\begin{ldescription}
\item[\code{Muestra}] Muestra de la que extraer las columnas indicadas.

\item[\code{Tipologia}] Tipologia del caso a estimar
\end{ldescription}
\end{Arguments}
\inputencoding{utf8}
\HeaderA{Tasacion}{Genera una tasacion}{Tasacion}
%
\begin{Description}\relax
Genera una tasacion a partir de 3 archivos csv ubicados en el directorio indicado:
- El caso a estimar (CasoEstimacion.csv)
- Los parametros del modelo (ParametrosCalculo.csv)
- La muestra de tasaciones de la zona (DatosMuestra.csv)
\end{Description}
%
\begin{Usage}
\begin{verbatim}
Tasacion(Directorio)
\end{verbatim}
\end{Usage}
%
\begin{Arguments}
\begin{ldescription}
\item[\code{Directorio}] 
Directorio en el que estan ubicados los archivos CasoEstimacion.csv, ParametrosCalculo.csv y DatosMuestra.csv

\end{ldescription}
\end{Arguments}
%
\begin{Details}\relax
Cabecera de los archivos csv (separador es el punto y coma)

CasoEstimacion.csv
-------------------
V005;FechaDeCierre;NumeroDeExpediente;ValorUnitarioDeMercado;NumeroDeBanos;NumeroDeDormitorios;SuperficieConstruidaDeLaVivienda;CosteDeConstruccionBrutoUnitarioVivienda;PorcentajeDepreciacionVivienda;NumeroDeAscensores;Tipologia;SuperficieDeLaParcela;DatosDeLaPlanta;NumeroDePlanta;ExteriorInterior

DatosMuestra.csv
-----------------
FechaDeCierre;NumeroDeExpediente;ValorUnitarioDeMercado;NumeroDeBanos;NumeroDeDormitorios;SuperficieConstruidaDeLaVivienda;CosteDeConstruccionBrutoUnitarioVivienda;PorcentajeDepreciacionVivienda;NumeroDeAscensores;Tipologia;SuperficieDeLaParcela;DatosDeLaPlanta;NumeroDePlanta;TipoZona;idZona;CodZona;Distancia;CodZonaReferencia;ExteriorInterior

ParametrosCalculo.csv
---------------------
Tipologia;Tipo;Fecha;x1;x2;x3;x4;x5;cte;media zona;SinAscensor;EsPlantaBaja;EsSemiSotano;EsInterior;EsAtico;NDePlanta

La tasacion se graba en un archivo Tasacion.csv en el mismo directorio Sin cabecera con la siguiente informacion:

Valor Estimado
Valor Estimado Inferior
Valor Estimado Superior
Valor Estimado Inferior (en tanto por 1)
Valor esitmado Superior (en tanto por 1)
\end{Details}
%
\begin{Value}
Devuelve la siguiente informacion

Valor Estimado
Valor Estimado Inferior
Valor Estimado Superior
Valor Estimado Inferior (en tanto por 1)
Valor esitmado Superior (en tanto por 1)
\end{Value}
\inputencoding{utf8}
\HeaderA{TransformarVariables}{Devuelve el logaritmo de las columnas recibidas}{TransformarVariables}
%
\begin{Usage}
\begin{verbatim}
TransformarVariables(Muestra)
\end{verbatim}
\end{Usage}
%
\begin{Arguments}
\begin{ldescription}
\item[\code{Muestra}] Muestra de la que devolver el logaritmo
\end{ldescription}
\end{Arguments}
\printindex{}
\end{document}
